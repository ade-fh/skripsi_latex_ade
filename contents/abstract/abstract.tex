\textit{
	Product piracy or counterfeiting is an illegal act carried out with the aim of gaining profit by imitating or copying patented or copyrighted products. In 2016, the trade of counterfeit and pirated goods reached \$509 billion (3.3\% of total global trade). To address this problem, manufacturers can use secure QR codes (SQR) affixed to their products. SQR are ordinary QR codes equipped with a copy detection pattern (CDP), which is a matrix pattern that is resistant to copying. Through the print and scan process, the quality of the CDP degrades, making it possible to distinguish between the original and counterfeit CDPs.
}

\textit{
	To obtain the CDP object used for the authentication process, four QR code corner points can be used, which will then be transformed and cropped to the predetermined CDP size. However, the quality of CDP localization using four points is not good if the SQR is captured in less-than-ideal conditions, such as when the SQR is affixed to a curved surface. This research aims to address this problem by using eight ArUco markers around the CDP object to improve the quality of CDP localization in less-than-ideal conditions.
}

\noindent\textbf{Keywords} : \emph{Copy Detection Pattern}, \emph{Secure QR Code}, \emph{ArUco Marker}, \emph{AutoML}, \emph{AutoGluon}, \emph{Anti-Counterfeiting}