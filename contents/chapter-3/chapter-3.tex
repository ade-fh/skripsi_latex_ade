\chapter{Metode Penelitian}

Bab ini menjelaskan metode atau cara yang digunakan dalam penelitian ini untuk 
mencapai maksud dan tujuan seperti yang tertulis dalam sub-bab 1.3 [jika diinginkan, kalian dapat menuliskan Kembali tujuan penelitian yang ingin dicapai di sini].

\section{Alat dan Bahan Tugas akhir (Opsional)}

\subsection{Alat Tugas akhir}



Alat-alat yang digunakan pada tugas akhir ini berupa perangkat keras maupun perangkat lunak sebagai sarana pendukung antara lain. Kemukakan secara detail sesuai dengan kebutuhan tugas akhir dan juga tambahkan spesifikasi minumum sehingga peneliti lain yang hendak melakukan hal yang sama bisa melakukannya :

\begin{enumerate}
	\item \textit{Notebook} dengan spesifikasi minumum sistem operasi Windows 8, \textit{processor} Intel Core i3 2330M CPU @ 2,2 GHz, memori 4GB DDR3, grafis NVIDIA GeForce GT 610 (4GB), hardisk 500GB. Pada tugas akhir ini digunakan Windows 10, Intel Core i7 4570M CPU, Memori 4GB DDR 3, grafis Intel HD4300. 
	\item \textit{Smartphone} dengan spesifikasi tipe minimum, OS Android OS v4.1.2 (Jelly Bean),CPU Dual-core 800 MHz, GPU Mali-400, Internal 4 GB, 768 MB RAM. Pada tugas akhir ini digunakan ....
	\item \textit{Game creation platform} versi 3.3.2 untuk Stencyl dan Construct2.
	\item CORELDRAW X7, Tiled dan GIMP 2
\end{enumerate}

\subsection{Bahan Tugas akhir}

Bahan tugas akhir adalah segala sesuatu yang bersifat fisik atau digital yang digunakan untuk kebutuhan tugas akhir. Bahan tugas akhir dapat berupa:

\begin{enumerate}
	\item Bahan habis pakai. Bahan yang digunakan untuk tugas akhir. Sebagai contoh 
	mungkin dibutuhkan kertas transparansi, baterai, atau yang lain 
	\item Bahan yang berupa data atau informasi yang menjadi dataset tugas akhir. Dataset tugas akhir dapat berupa:
\end{enumerate}
\begin{itemize}
	\item Dataset pihak lain yang diperoleh dengan izin atau dalam lisensi yang diizinkan untuk digunakan secara langsung 
	\item Dataset pihak pertama yang disusun sendiri melalui quisioner, observasi, atau interview 
	\item Dokumen panduan yang mengacu pada standar, hasil tugas akhir, atau artikel yang disitasi dan digunakan.
\end{itemize}


\section{Metode yang Digunakan}

Bagian ini membahas metode atau cara yang akan digunakan dalam penelitian, tahapan 
penerapan metode, dan desain penelitian (misalnya apakah penelitian akan menggunakan 
eksperimen di Laboratorium atau di lapangan, misalkan saja penelitian biomedis atau 
penelitian alat ukur hama yang dapat dilakukan di laboratorium ataupun di lapangan, atau menggunakan metode survei (misalnya untuk teknologi Informasi), studi kasus, atau analisis dengan perangkat lunak (ETAP, LTSpice, dst), atau \textit{prototyping} (pembuatan perangkat keras).

Bagian ini juga membahas bagaimana data [akan] dianalisis, apakah dengan membandingkan keluaran beberapa alat ukur, membandingkan dengan standar atau bagaimana.

\section{Alur Tugas Akhir}

Menguraikan prosedur yang akan digunakan dan jadwal atau alur penyelesaian setiap 
tahap. Alur penelian ini dapat disajikan dalam bentuk diagram. Diagram dapat disusun dengan aturan yang baik semisal menggunakan \textit{flowchart}. Aturan dan tutorial pembuatan \textit{flowchart} dapat dilihat di \textcolor{blue}{http://ugm.id/flowcharttutorial}. Setelah menggambarkannya, penulis wajib menjelaskan langkah-langkah setiap alur tugas akhir dalam sub bab tersendiri sesuai dengan kebutuhan.

\section{Etika, Masalah, dan Keterbatasan Penelitian (Opsional))}

Bagian ini membahas pertimbangan etis penelitian dan [potensi] masalah serta
keterbatasannya. Jika menyangkut penelitian dengan makhluk hidup, maka dibutuhkan adanya \textit{ethical clearance}, di bagian ini hal itu akan dibahas. Demikian juga tentang keterbatasan ataupun masalah yang akan timbul.
