\chapter{Kesimpulan dan Saran}

\section{Kesimpulan}
\begin{itemize}
      \item Model SQR yang dirancang penulis (Kode QR dengan \emph{watermark}) dapat diautentikasi dengan akurasi sempurna 100\% (pada \emph{dataset} yang dilokalisasi
            dengan penanda ArUco)
      \item Selain model SQR, keluaran dari penelitian ini adalah parameter P\&S untuk pembuatan \emph{dataset} SQR.
      \item Dari uji signifikansi menggunakan metode \emph{T-test} antara data grup orisinal dengan palsu, pada \emph{dataset} CDP yang dilokalisasi dengan 8 titik
            memiliki signifikansi tinggi pada seluruh koefisien jaraknya, sedangkan pada \emph{dataset} CDP yang dilokalisasi dengan 4 titik, hasilnya tidak terlalu
            signifikan, terlihat dari banyaknya \emph{overlap} pada plot distribusinya.
            %     \item Hasil uji signifikansi menggunakan metode \emph{T-test} antara jarak CDP 2 level dan 4 level dengan \emph{template}, baik orisinal maupun palsu, menunjukkan
            %           perbedaan signifikan. Dibandingkan dengan CDP 2 level, CDP 4 level orisinal dan palsu memiliki jarak yang lebih dekat dengan \emph{template}.
            %     \item Dari uji signifikansi menggunakan metode \emph{T-test}, fitur spasial memisahkan CDP orisinal dan palsu secara distribusi dengan baik. Fitur histogram baik
            %           intuk memisahkan CDP 2 dan 4 level. Adapun fitur \emph{dct\_canberra} yang juga baik dalam memisahkan CDP orisinal dengan palsu.
      \item Dari pengujian pembuatan model autentikasi menggunakan fitur tunggal pada \emph{dataset} CDP yang dilokalisasi menggunakan 8 titik, fitur \emph{sp\_corr} dan
            \emph{dct\_canberra} menjadi fitur fitur terbaik yang mendapatkan akurasi sempurna 100\%, sedangkan pada \emph{dataset} CDP yang dilokalisasi menggunakan 4
            titik, akurasi tertingginya adalah 92,5\% menggunakan fitur \emph{his\_canberra}.
      \item Dari pengujian pembuaatan model autentikasi menggunakan multi fitur, pada \emph{dataset} CDP yang dilokalisasi menggunakan 8 titik memiliki akurasi tertinggi
            100\% menggunakan model XGBoost, sedangkan pada \emph{dataset} CDP yang dilokalisasi menggunakan 4 titik, akurasi tertingginya adalah 98,33\%.
      \item AutoML dari AutoGluon dapat membuat model autentikasi dengan baik untuk kasus klasifikasi biner CDP orisinal dan palsu, baik untuk CDP 2 maupun 4 level.
\end{itemize}

\section{Saran}
\begin{itemize}
      \item Model SQR dan parameter P\&S ini dapat digunakan untuk pembuatan \emph{dataset} berikutnya dengan jumlah yang lebih besar.
      \item Dibutuhkan pembuatan \emph{dataset} SQR dengan lingkungan pemotretan yang berbeda, semisal dipotret langsung di ruang terbuka dengan berbagai kondisi, namun
            hasilnya harus tetap baik (fokus dan tidak kabur) dengan menggunakan parameter konfigurasi kamera yang penulis dapatkan.
      \item Parameter konfigurasi kamera yang telah penulis dapatkan dapat digunakan untuk pengembangan aplikasi \emph{mobile} untuk autentikasi SQR.
            %     \item Menurut penulis, CDP dengan 4 level \emph{grayscale} lebih baik untuk diterapkan dalam pengembangan SQR karena secara teoritis lebih sulit untuk direpilkasi.
            %           Apabila ukuran CDP adalah 100x100 piksel, untuk mereplikasi secara tepat, diperlukan $4^{10000}$ kombinasi yang harus dicoba. Keamanan juga dapat ditingkatkan
            %           dengan mengubah keempat level menjadi dinamis. Saat ini level \emph{grayscale} yang digunakan adalah tetep, yaitu [0, 85, 170, 255]. Level \emph{grayscale}
            %           tersebut dapat diubah menjadi dinamis seperti [0, 45, 160, 220], [20, 80, 180, 255], dan seterusnya secara acak. Hal tersebut kemungkinan dapat meningkatkan
            %           keamanan dari CDP, namun masih perlu diuji melalui penelitian-penelitian selanjutnya. Alasan lain adalah, pada kondisi data CDP yang kurang baik, dalam hal ini
            %           hasil CDP yang dilokalisasi tanpa menggunakan ArUco \emph{marker}, CDP 4 level dapat diautentikasi dengan akurasi yang lebih tinggi daripada CDP 2 level.
      \item Penelitian tentang penyerangan estimasi CDP 4 level perlu dilakukan untuk mengetahui efektivitas dan ketahanan CDP 4 level.
      \item Fitur-fitur yang memiliki akurasi tinggi dalam model fitur tunggal dan memiliki signifikansi distribusi tinggi dalam membedakan CDP orisinal dan palsu seperti
            \emph{sp\_corr}, \emph{sp\_cosine}, \emph{sp\_euclidean}, \emph{sp\_canberra}, dan \emph{dct\_canberra} dapat dijadikan fitur utama yang digunakan dalam
            pembuatan model autentikasi multi fitur.
      \item Dibutuhkan pustaka untuk pembacaan data kode QR yang lebih cepat dari pustaka yang dimiliki Python (OpenCV), misalnya pustaka pembaca QR \emph{native} dari
            Java Android.
\end{itemize}