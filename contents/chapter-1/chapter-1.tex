\chapter{Pendahuluan}

\section{Latar Belakang}
Pembajakan produk atau yang dikenal juga sebagai tindakan pemalsuan, merupakan suatu tindakan ilegal yang dilakukan dengan tujuan memperoleh keuntungan dengan
cara meniru atau menyalin produk asli yang sudah dipatenkan atau memiliki hak cipta. Pembajakan produk semakin marak di era digital dan globalisasi, seiring
dengan perkembangan teknologi. Di era digital, pembajakan produk semakin mudah dilakukan dengan memanfaatkan internet dan teknologi digital. Sementara itu,
globalisasi mempermudah transportasi dan distribusi produk palsu dari satu negara ke negara lain. Pembajakan produk juga menyebabkan kerugian ekonomi yang
signifikan bagi produsen dan pemilik hak merek, serta dapat membahayakan keselamatan konsumen. Hal ini terjadi karena pembajakan produk dapat merusak citra
perusahaan, mengurangi pendapatan, serta merugikan konsumen yang membeli produk palsu yang seringkali berkualitas rendah dan dapat membahayakan diri, baik
secara langsung maupun tidak langsung \cite{BASCAP2016}.

Teknologi percetakan dan pemindai digital telah mengalami perkembangan yang pesat selama beberapa dekade terakhir. Namun, kemajuan ini tidak hanya digunakan
untuk kegiatan positif juga dimanfaatkan oleh pelaku pembajakan untuk memproduksi produk-produk bajakan. Dengan kemampuan teknologi percetakan dan pemindai
yang semakin canggih, pembajakan produk menjadi lebih mudah, lebih cepat, dan lebih murah \cite{HILL2007DIGITAL}.

Printer 2D beresolusi tinggi dan Printer 3D, memungkinkan pembuat produk bajakan untuk membuat produk-produk dengan kualitas hampir sama dengan produk asli.
Pemindai 3D juga pembuat produk bajakan untuk menyalin produk asli hingga detail terkecil dengan cepat dan mudah. Selain itu, teknologi digital seperti desain
grafis dan \emph{software} pemodelan juga memudahkan pelaku pembajakan dalam membuat desain dan cetakan produk tanpa harus membeli hak cipta atau paten produk
tersebut \cite{DEPOORTER2013INTELECTUAL}.

\emph{Copy detection pattern} (CDP) merupakan sebuah matriks pola dengan ukuran tertentu yang peka terhadap salinan yang akan terdegradasi kualitasnya akibat proses \emph{print} \&
\emph{scan} (P\&S). Jumlah informasi yang hilang atau terdegradasi digunakan untuk membedakan CDP asli atau palsu. CDP ini dapat diimplementasikan ke dalam kode QR.
Kode QR yang dilengkapi CDP dalam penelitian ini nantinya akan disebut dengan \emph{secure QR code} (SQR). SQR dapat diimplementasikan pada produk untuk
mengautentikasi keaslian produk.

% Penelitian sebelumnya menggunakan CDP dengan 2 level \emph{grayscale}. Hasilnya, CDP tersebut dapat digunakan untuk
% mengautentikasi SQR orisinal dan palsu dengan sangat baik. Namun, menurut penulis CDP dengan 2 level \emph{grayscale} mudah untuk disalin karena untuk menyalin
% tiap-tiap piksel dalam CDP, hanya ada 2 variasi yang mungkin, yaitu hitam dan putih.

% Seseorang dapat menyalin CDP asli melalui proses P\&S dan mengestimasi pola aslinya, yaitu \emph{template} atau CDP versi digitalnya. Matriks \emph{bit error
%       rate} (BER) digunakan untuk mengukur signifikansi antara CDP 2 level hasil estimasi dengan CDP 2 level \emph{template}. Dari penelitian sebelumnya yang
% menggunakan CDP dengan 2 level \emph{grayscale} didapatkan nilai BER minimal adalah 19,92\% \cite{PICARDCANCOPYDETECTIONPATTERN}. Penelitian selanjutnya yang
% membuat model estimasi menggunakan parameter yang lebih sedikit mendapatkan nilai BER minimal 9,91\% \cite{penelitianpaksyukron}. Semakin kecil nilai BER, maka
% semakin mungkin sebuah CDP 2 level diestimasi. Dari permasalahan tersebut, penulis mencoba untuk membuat model SQR yang dapat diautentikasi menggunakan CDP 4
% level \emph{grayscale} dan membandingkannya dengan CDP 2 level \emph{grayscale}. SQR yang didesain oleh penulis nantinya akan dilengkapi dengan sebuah
% \emph{watermark} yang didalamnya terdapat CDP dan juga delapan penanda ArUco.

\section{Rumusan Masalah}
Dari permasalahan yang didapatkan dari penelitian sebelumnya, yaitu hasil lokalisasi CDP yang kurang baik jika hanya menggunakan 4 titik acuan (4 titik sudut
kode QR), penulis mencoba menggunakan 8 penanda ArUco \cite{arucoopencv} di sekitar objek CDP untuk meningkatkan kualitas lokalisasi CDP pada kondisi SQR yang kurang ideal
(misalnya ada lekukan pada sisi SQR karena SQR ditempelkan pada permukaan tabung).

% Dari penelitian sebelumnya tentang penyerangan estimasi terhadap CDP 2 level, didapatkan CDP hasil estimasi memiliki nilai BER 19,92\% dan 9,91\%. Hal tersebut
% menunjukkan menunjukkan bahwa CDP 2 level tidak cukup aman untuk digunakan. Selain itu, lokalisasi CDP menggunakan 4 titik (4 titik sudut kode QR) hasilnya
% kurang baik jika hasil pemotretan kurang ideal (misalnya ada \emph{bending} pada sisi SQR karena SQR ditempelkan pada permukaan tabung). Penelitian ini mencoba
% untuk mengatasi permasalahan tersebut dengan menggunakan CDP 4 level yang secara teoritis lebih sulit untuk direplikasi karena memiliki 4 level gradasi warna
% yang berbeda untuk diestimasi dan juga penggunaan penanda ArUco untuk meningkatkan kualitas lokalisasi CDP pada kondisi yang kurang ideal.
% Berdasarkan latar belakang yang sudah dijabarkan di atas, maka rumusan masalah dalam penelitian ini meliputi:
% \begin{enumerate}
%       \item Apakah hasil lokalisasi CDP menggunakan 8 penanda ArUco (8 titik) lebih baik daripada tanpa menggunakan penanda ArUco (4 titik sudut QR)?
%       \item Bagaimana karakteristik fitur jarak yang didapatkan dari CDP dengan 2 dan 4 level \emph{grayscale}?
%       \item Apakah CDP 4 level \emph{grayscale} dapat diautentikasi dengan lebih baik oleh model dibandingkan CDP 2 level \emph{grayscale}?
% \end{enumerate}

\section{Tujuan Penelitian}
Tujuan dari penelitian ini adalah untuk mengetahui efektivitas penggunaan penanda ArUco di sekitar CDP untuk membantu melokalisasi CDP yang digunakan dalam
autentikasi SQR orisinal atau palsu.
% \begin{enumerate}
%       % \item Membuat model SQR yang dilengkapi \emph{watermark} yang dapat diautentikasi oleh model.
%       % \item Mengetahui nilai parameter P\&S dalam pembuatan SQR (konfigurasi kamera dan pencetakan).
%       \item Mengetahui efektivitas penggunaan penanda ArUco dalam melokalisasi objek CDP.
%             % \item Membuat model untuk melakukan autentikasi SQR orisinal dan palsu.
%             % \item Membandingkan performa model dalam mengautentikasi CDP dengan 2 dan 4 level \emph{grayscale}.
% \end{enumerate}

\section{Batasan Penelitian}
Beberapa batasan yang penulis gunakan dalam penelitian ini antara lain:
\begin{enumerate}
      \item Objek Penelitian: Analisis penggunaan penanda ArUco untuk melokalisasi objek CDP yang digunakan untuk autentikasi SQR orisinal atau palsu.
      \item Metode Penelitian: Analisis dilakukan dengan eksperimen yang mencakup verifikasi parameter pembuatan model SQR, pembuatan \emph{dataset} SQR orisinal dan
            palsu, serta pembuatan model untuk mengautentikasi SQR. Hasil dari penelitian ini adalah mengetahui efektivitas penanda ArUco dalam melokalisasi objek CDP.
            Hasil tersebut diperoleh dari analisis data yang didapatkan melalui eksperimen.
      \item Waktu dan Tempat Penelitian:
            \begin{itemize}
                  \item Waktu penelitian: Agustus 2022 s.d. April 2023
                  \item Tempat penelitian: Ruang rispro, laboratorium informatika, laboratorium jaringan dan komputer, dan rumah penulis.
            \end{itemize}
      \item Populasi dan Sampel: Populasi adalah seluruh \emph{dataset} SQR yang ada. Sampel penelitian adalah \emph{dataset} SQR yang dibuat dan digunakan oleh penulis
            dalam penelitian.
            % Populasi penelitian berupa seluruh \emph{dataset} SQR yang dibuat dan digunakan dalam penelitian. Sampel penelitian berupa sebagian
            % \emph{dataset} SQR yang digunakan untuk penentuan parameter dan pembuatan model klasifikasi biner.
      \item Variabel: Variabel bebas meliputi \emph{dataset} SQR yang dibuat oleh penulis, variabel kontrolnya adalah \emph{environment} dan parameter yang digunakan dalam
            pemotretan \emph{dataset}, sedangkan variabel terikatnya adalah hasil lokalisasi CDP dan hasil klasifikasi biner oleh model autentikasi (SQR orisinal atau
            palsu).
      \item Hipotesis:
            \begin{itemize}
                  \item Hasil lokalisasi CDP menggunakan delapan penanda ArUco (8 titik) lebih akurat dibandingkan tanpa menggunakan penanda ArUco (4 titik sudut kode QR).
                        % \item CDP dengan 4 level \emph{grayscale} dapat diautentikasi dengan akurasi yang lebih baik dibandingkan CDP dengan 2 level \emph{grayscale}.
            \end{itemize}
      \item Keterbatasan Penelitian:
            \begin{itemize}
                  \item \emph{Printer} yang digunakan untuk mencetak \emph{dataset} SQR seragam.
                  \item Pemotretan \emph{dataset} SQR dilakukan dalam kondisi pencahayaan yang baik, berasal dari \emph{flash smartphone}.
                  \item Kamera, konfigurasi kamera, sudut dan kondisi pemotretan \emph{dataset} adalah tetap (menggunakan bantuan boks, sehingga jarak objek dengan kamera tetap).
                        % \item Parameter yang digunakan untuk menilai efektivitas CDP 2 dan 4 level pada SQR adalah sebatas akurasi klasifikasi biner oleh model. Selain itu, diasumsikan
                        %       bahwa CDP dengan 4 level \emph{grayscale} lebih sulit diestimasi oleh penyerang.
                  \item Penggunaan penanda ArUco memiliki kelemahan, yaitu model deteksi akan sulit mendeteksi penanda ArUco jika ukurannya sangat kecil.
            \end{itemize}
            % \item \textit{Printer} yang digunakan untuk mencetak kumpulan data SQR seragam.
            % \item Pemotretan kumpulan data SQR dilakukan dalam kondisi pencahayaan yang baik yang berasal dari \emph{flash} hp.
            % \item Kamera, konfigurasi kamera, sudut dan kondisi pengambilan gambar yang digunakan dalam melakukan pemotretan adalah tetap.
\end{enumerate}

\section{Manfaat Penelitian}
\begin{itemize}
      \item Bagi peneliti dan pengembang SQR selanjunya, mereka dapat menggunakan model SQR, parameter \emph{print} \& \emph{scan} (P\&S), dan model autentikasi
            (klasifikasi CDP orisinal dan palsu) yang memiliki akurasi terbaik.
      \item Bagi penulis, penelitian ini dapat menambah wawasan, ilmu dan pengetahuan dalam pembuatan tulisan ilmiah, khususnya pada topik spesifik seperti keamanan
            digial, pengolahan citra gambar, dan pembelajaran mesin.
      \item Bagi pelaku bisnis, penerapan SQR dalam produk dapat mengurangi risiko dan melindungi produk dari pembajakan.
\end{itemize}
% Dengan dilakukannya penelitian ini, pengembang SQR dapat menggunakan model SQR, parameter P\&S, dan model yang memiliki hasil klasifikasi biner terbaik
% (autentikasi CDP orisinal dan palsu). Selain itu, penggunaan penanda ArUco diharapkan mampu melokalisasi CDP dalam SQR dengan lebih cepat dan akurat
% dibandingkan tanpa menggunakan penanda ArUco. Bagi peneliti, penelitian ini dapat menambah wawasan, ilmu dan pengetahuan dalam pembuatan tulisan ilmiah,
% khususnya pada topik keamanan digital, pengolahan citra gambar, dan pembelajaran mesin. Bagi pelaku bisnis, penerapan SQR dalam produk dapat membantu mereka
% dalam melindungi produk dari pembajakan.

\section{Sistematika Penulisan}
\noindent
\textbf{BAB I : PENDAHULUAN}

\noindent Pada bab ini dijelaskan latar belakang, rumusan masalah, tujuan penelitian, batasan penelitian, manfaat penelitian, dan sistematika penulisan.\\

\noindent
\textbf{BAB II : TINJAUAN PUSTAKA DAN LANDASAN TEORI}

\noindent Pada bab ini dijelaskan teori-teori dan penelitian terdahulu yang digunakan sebagai acuan dan dasar dalam penelitian.\\

\noindent
\textbf{BAB III : METODOLOGI PENELITIAN}

\noindent Pada bab ini dijelaskan metode yang digunakan dalam penelitian meliputi alat dan bahan, metode, serta tahapan dan alur penelitian.\\

\noindent
\textbf{BAB IV : HASIL DAN PEMBAHASAN}

\noindent Pada bab ini dijelaskan hasil penelitian dan pembahasannya.\\

\noindent
\textbf{BAB V : KESIMPULAN DAN SARAN}

\noindent Pada bab ini ditulis kesimpulan akhir dari penelitian dan saran untuk pengembangan penelitian selanjutnya.\\